\documentclass[a4paper, 10pt]{paper}

\usepackage[UTF8]{ctex}
\usepackage{geometry}
\usepackage{minted}
\usepackage{fontspec}
\usepackage{color}

\newcommand{\cpp}[1]{\inputminted[frame=single, linenos=true]{cpp}{#1}}

\geometry{a4paper, scale=0.8}
\setmonofont[Path=C:/Users/21988/AppData/Local/Microsoft/Windows/Fonts/]{JetBrainsMono-Medium.ttf}
\usemintedstyle{autumn}
\setminted{fontsize=\small, breaklines=true, breakautoindent=true}

\begin{document}

    \title{算法竞赛模板}
    \author{by ChaomengOrion}
    \maketitle
    {\color{cyan}最后修改于\underline{\today}}

    \tableofcontents

    \section{前期准备}
        \subsection{Cpp文件一键编译测试}
        Windows - \textbf{build.bat}
        \inputminted[frame=single, linenos=true]{bat}{../build.bat}

        \subsection{Cpp模板}
        注意题目是不是多组样例
        \cpp{../template.cpp}

    \section{算法}
        \subsection{快速幂}
        \cpp{../快速幂.cpp}

        \subsection{DP}
        \subsubsection{LIS}
        \cpp{../LIS.cpp}
        \subsubsection{LIS}
        \cpp{../LCS.cpp}
        \subsubsection{01背包}
        \cpp{../0_1_背包.cpp}
        \subsubsection{完全背包}
        \cpp{../完全背包.cpp}
        \subsubsection{多重背包}
        \cpp{../多重背包.cpp}

        \subsection{二维差分}
        \begin{minted}[frame=single, linenos=true]{cpp}
void best()
{
    int N, M; std::cin >> N >> M;
    std::vector map(N + 1, std::vector<int>(N + 1, 0));
    std::vector diff(N + 2, std::vector<int>(N + 2, 0));
    for (int i = 1; i <= N; i++) for (int j = 1; j <= N; j++) std::cin >> map[i][j];

    while (M--) {
        int x1, x2, y1, y2;
        std::cin >> x1 >> y1 >> x2 >> y2;
        diff[x1][y1]++;
        diff[x1][y2 + 1]--;
        diff[x2 + 1][y1]--;
        diff[x2 + 1][y2 + 1]++;
    }
//  修改
//  ###+@@@-
//  ###@@@@#
//  ###-###+
    for (int i = 1; i <= N; i++)
        for (int j = 1; j <= N; j++)
            diff[i][j] += diff[i - 1][j];

    for (int i = 1; i <= N; i++)
        for (int j = 1; j <= N; j++)
            diff[i][j] += diff[i][j - 1];

    for (int i = 1; i <= N; i++) {
        for (int j = 1; j <= N; j++) {
            std::cout << diff[i][j] + map[i][j] << ' ';
        }
        std::cout << '\n';
    }
}
        \end{minted}

        \subsection{并查集}
        \begin{minted}[frame=single, linenos=true]{cpp}
void solve()
{
    int N, M;
    std::cin >> N >> M;
    std::vector<int> fa(N + 1);
    for (int i = 0; i <= N; i++) {
        fa[i] = i;
    }
    auto find = [&](auto&& find, int x) -> int {
        return x == fa[x] ? x : fa[x] = find(find, fa[x]);
    };

    while (M--) {
        int Z, X, Y;
        std::cin >> Z >> X >> Y;
        if (Z == 1) {
            fa[find(find, Y)] = fa[find(find, X)];
        } else if (Z == 2) {
            std::cout << (fa[find(find, Y)] == fa[find(find, X)] ? 'Y' : 'N') << std::endl;
        }
    }
}
        \end{minted}

        \subsection{图论}
        \subsubsection{最短路(Dijkstra)}
        \cpp{../dijkstra.cpp}

        \subsubsection{最短路(SPFA)}
        \cpp{../SPFA.cpp}

        \subsubsection{最小生成树}
        \cpp{../dijkstra.cpp}

        \subsubsection{最近公共祖先LCA}
        \cpp{../最近公共祖先_LCA.cpp}

    \section{数据结构}
        \subsection{树状数组}
        \cpp{../BIT.cpp}
        实现区间查询区间修改
        \begin{minted}[frame=single, linenos=true]{cpp}
void solve()
{
    int N, M;
    std::cin >> N >> M;
    std::vector<int> diff(N + 1, 0);
    std::vector<i64> diff2(N + 1, 0);
    int last = 0;
    for (int i = 1; i <= N; i++) {
        int temp;
        std::cin >> temp;
        diff[i] = temp - last;
        diff2[i] = 1LL * (i - 1) * diff[i];
        last = temp;
    }

    BIT<int> tr1(diff);
    BIT<i64> tr2(diff2);
    // pre[i] = k*Σ(D[i]) - Σ((i-1)*D[i])
    while (M--) {
        int op;
        std::cin >> op;
        if (op == 1) {
            int x, y, k;
            std::cin >> x >> y >> k;
            tr1.add(x, k);
            tr1.add(y + 1, -k);
            tr2.add(x, 1LL * (x - 1) * k);
            tr2.add(y + 1, -1LL * y * k);
        } else if (op == 2) {
            int x, y;
            std::cin >> x >> y;
            i64 sum1 = 1LL * y * tr1.query(y) - tr2.query(y);
            i64 sum2 = 1LL * (x - 1) * tr1.query(x - 1) - tr2.query(x - 1);
            std::cout << sum1 - sum2 << std::endl;
        }
    }
}
        \end{minted}

        \subsection{ST表}
        \cpp{../ST.cpp}

        \subsection{线段树}
        \cpp{../SegTree.cpp}

    \section{计算几何}
        \subsection{Point}
        \cpp{../Point.cpp}

    \section{数学相关}
        \subsection{模运算}
        \begin{minted}[frame=single, linenos=true]{cpp}
i64 add(i64 a, i64 b, i64 p) { // 加
    return (a % p + b % p) % p;
}

i64 sub(i64 a, i64 b, i64 p) { // 减
    return (a % p - b % p) % p;
}

i64 mul(i64 a, i64 b, i64 p) { // a > p 乘
    a %= p;
    b %= p;
    i64 ans = 0;
    while (b > 0) {
        if (b & 1) {
            ans += a;
            ans %= p;
        }
        a <<= 1;
        a %= p;
        b >>= 1;
    }
    return ans;
}
        \end{minted}

        \subsection{素数筛\&欧拉函数}
        \cpp{"../素数筛&欧拉函数.cpp"}

    \section{杂项}
        \subsection{快速读入}
        \begin{minted}[frame=single, linenos=true]{cpp}
inline int read()
{
    int x = 0, sgn = 1;
    char ch = getchar();
    while (ch < '0' || ch > '9') {
        if (ch == '-') sgn = -1;
        ch = getchar();
    }
    while (ch >= '0' && ch <= '9') {
        x = x * 10 + ch - '0';
        ch = getchar();
    }
    return x * sgn;
}
        \end{minted}

\end{document}